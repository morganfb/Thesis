%!TEX root = ../dissertation.tex
% the abstract

The statistical field of causal inferences seeks to develop methods that under certain assumptions use observational data to estimate causal effects without the need for costly and time-intensive randomized controlled trials.  Two such methods, the g-formula and a doubly robust estimator, are investigated and compared in this thesis, particularly with time-varying treatment regimens.  The g-formula is computationally intensive and has the potential to be biased by the incorrect specification of models for outcome and confounders.  The doubly robust method is stronger than the g-formula at detecting underlying correlation between outcome and treatment status.  While the doubly robust method has slightly higher variance than the g-formula, it is much more robust to errors in model specification.  It is also even more robust than initially thought, allowing for significant misspecification of models on the part of the researcher, as shown herein.   The doubly robust method is not only multiply robust, but having never before been implemented in the programming language Python, this new program is also significantly more computationally efficient.  Furthermore, this first ever implementation in Python is expected to decrease many barriers to use for the scientific community, opening pathways to countless discoveries from massive amounts of data currently available and underutilized.  

%% Need to say first to implement in python 