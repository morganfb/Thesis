%!TEX root = ../dissertation.tex
% the abstract

The field of causal inference seeks to develop methods that under certain assumptions use observational data to estimate causal effects without the need for costly and time-intensive randomized controlled trials.  Two such methods, the g-formula and a doubly robust estimator, are investigated and compared in this thesis, particularly with time-varying treatment regimens.  The g-formula can become biased by the incorrect specification of parametric models and under a treatment effect.  The doubly robust method is stronger than the g-formula at detecting underlying correlation between outcome and treatment but is slightly less efficient than the g-formula.  While the doubly robust method has slightly higher variance than the g-formula, it is much more robust to errors in model specification.  It is also even more robust than initially thought, allowing for significant misspecification of models on the part of the researcher, as shown herein.   The doubly robust method is not only multiply robust, but having never before been implemented in the programming language Python, it is now also significantly more computationally efficient than previously seen.  In short, this multiple robustness and new implementation in Python are expected to decrease many barriers to use for the academic community, opening pathways to countless discoveries from massive data currently underutilized from various fields, including medicine, finance, and the social sciences.  
