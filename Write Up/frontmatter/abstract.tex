%!TEX root = ../dissertation.tex
% the abstract

The statistical field of causal inferences seeks to develop methods which are able to prove causation from observational data without the need for costly and time intensive randomized trials.  Two such methods, the g-formula and doubly robust estimation, are investigated and compared in this thesis, particularly with time-varying treatment regimens.  The g-formula is computationally intensive and has the potential to be biased by the incorrect specification of models.  However, with care on the part of the researcher and a high number of Monte Carlo simulations, a precise measurement of causal treatment effect can be gained.  Furthermore, the doubly robust method is stronger at detecting underlying correlation between outcome and treatment status than the g-formula.  On the other hand, the doubly robust is slightly less precise in its estimation of causal treatment effect, with slightly higher variance, but is much more robust to errors in model specification.  It is also even more robust than initially thought, allowing for significant misspecification of models on the part of the researcher, as shown herein.   This developed method is not only multiply robust, but its implementation in Python is also significantly more computationally efficient.  Furthermore, it is decreasing many barriers to use for the scientific community, opening pathways to countless imminent discoveries from the massive data currently available and underutilized.  