%!TEX root = ../dissertation.tex

\chapter{Methods}

\newthought{Lorem ipsum dolor sit amet}, 


\section{Data Creation} 


%% Testing under the null of full treatment, but using data on time varying treatment 
\section{Parametric G-formula} 
Similar to IP weighting, parametric estimates can be obtained for standardized estimates.  An efficient method for doing this is the generalization of standardization to time-varying treatments and confounders, coined the g-formula method by Robins in 1986.\cite{robins1986new, wright2015international, hernan_robins_2016}  The method can be used for fixed and time-varying treatments in longitudinal studies, and it seeks to estimate the average causal effect of treatment.  

which can be estimated as 
\begin{align} 
\mathbb{E}[Y^{\bar{a} = \bar{1}}] - \mathbb{E}[Y^{\bar{a} = \bar{0}}] 
\end{align} 
where the respective $\bar{a} = \bar{1}$ and $\bar{a} = \bar{0}$ signify constant treatment and no treatment over the entire time period.  

The g-formula seeks to calculate each standardized mean using the following, 
\begin{align} \label{eq:3} 
\mathbb{E}[Y^{\bar{a}= \bar{1}}] &= \sum_{l_i} \mathbb{E} \big[Y \mid  \overline{L}_{t}, \overline{A}_{t} \big]\cdot \prod_{k=0}^t P[L_k = l_k \mid \overline{L}_{k-1}, \overline{A}_{k-1}]
\end{align}
where $\overline{L}_k = \{L_{0} = l_0, L_{1} = l_1,  \cdots, \; L_{k} = l_k\}$ and $\overline{A}_k = \{a_{0} = 1, a_{1} = 1,  \cdots, \; a_{k} = 1\}$.  The equivalent formula can be derived for $ \mathbb{E}[Y^{\bar{a} = \bar{0}}]$.   In equation \ref{eq:3}, the summation term is 
%% come back here to discuss more about why you would use this method, how it was developed, details from Jamie's original paper, etc


\subsection{Protocol} 
The method is performed in several steps, as follows 
\begin{enumerate}
\item \underline{Expand the dataset}: Create two new simulated datasets, the first has all individuals under no treatment ($A=0$) and the second of which has all individuals treated ($A=1$).  Each of these new datasets has the same size as the original and the same ``individuals'', just changed values for $A$.  For these two datasets, delete the outcome values for $Y$ to treat it as a missing data.  

\item \underline{Create outcome models}: Create models for the outcome variable $Y$ and the covariates, $L_k$ at each time using the original dataset and the two datasets created in step 1 together.  The model for $Y$ is regressed on the treatment variable $A$ and the confounders, $L$.  Note that because the data in these two new datasets is missing for $Y$, they will not actually contribute to the model's parameters.  

In this case, the following models were chose for $Y \mid  \overline{A}_t, \overline{L}_t$ and $L_k \mid \overline{L}_{k-1}, \overline{A}_{k-1}$, 
\begin{align} 
\mathbb{E} \big[Y \mid \overline{A}_t, \overline{L}_t \big] &= \theta_{0} + \theta_1 A_{t} + \cdots + \theta_j A_0 + \theta_{j+1} L_t + \cdots + \theta_{j+k} L_0 \label{eq:2} \\ 
logit[L_k \mid \overline{L}_{k-1}, \overline{A}_{k-1}] &= \gamma_0 + \gamma_1 L_{k-1} + \gamma_2 L_{k-2} + \gamma_3 A_{k-1} + \gamma_4 A_{k-2} 
\end{align} 
A time lag of only two historical values was deemed sufficient for the model of $L_k$ because ... 
%% PUT SOME REASONS IN HERE DISCUSSING HOW THESE MODELS WERE CHOSEN... SHOULD I BE USING QUADRATIC TERMS OR INTERACTION TERMS??!?!?!?!


\item \underline{Predict}: Using the model created in step 2, predict the outcome $Y$ for the two new datasets created in step 1, conditioned on the given $A$ and $L$ values.  

Using equation \ref{eq:2} above, 

\item \underline{Standardization by averaging}:  Created a weighted average for $E[Y^{a=0}]$ from the first new dataset and one for $E[Y^{a=1}]$ from the second new dataset
\end{enumerate} 


\subsubsection{Variance Estimate} 
%% TALK THROUGH HOW THIS IS DONE USING BOOTSTRAPPING 


\section{Doubly Robust Estimation} 
The method of doubly robust estimation, as proposed by Bang and Robins \cite{bang2005doubly}, combines the two previously discussed methods of IP weighting and standardization.  

IP weighting and standardization techniques are expected to provide different answers, unless there are no models used to create estimates.\cite{hernan_robins_2016}  
IP weighting estimates $P[A=a \mid L =l]$ using $P[A =a \mid L= l]$, while standardization estimates $E[Y \mid A = a, L=l]$  
\newpage 

\begin{align*} 
Y &= \beta_0 + \beta_1 X \\ 
expit(Y) &= expit(\beta_0 + \beta_1 X) \\ 
\end{align*} 
