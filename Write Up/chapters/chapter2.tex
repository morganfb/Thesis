%!TEX root = ../dissertation.tex

\chapter{Methods}

\newthought{Two formal methods for estimating}, causal effect were used in this study: g-formula estimation and doubly robust estimation.  These two methods are developments of standardization and IP weighting as discussed in the previous section.  They were implemented and studied using simulated data according to the following methods.  

\section{Data Creation} 
For the purposes of this study, a data generating algorithm was created to provide consistent and easily accessible data for many simulations.  The data generated was time-varying and sequentially randomized according to the following schematic.  

\begin{figure}[h!]
  \begin{displaymath}
    \xymatrix{
    	& & & & & & \mathbf{Y} \\
        \mathbf{L_0} \ar[r] & \mathbf{A_0} \ar[r]  & \mathbf{ L_1} \ar[r]  & \mathbf{A_1} \ar@{.>}[r] & \mathbf{L_K} \ar[r] & \mathbf{A_K} &  \\ 
        & & & & & \ar[lllllu] \ar[lllu]  \ar[lu] \mathbb{U}  \ar[ruu]  & 
        }
\end{displaymath}
  \centering{\caption{Diagram of conditional dependencies in data generating process.}}
 \end{figure}


The algorithm to generate datasets is as follows for each individual, of which 1,000 were simulated in this study.  
\begin{enumerate} 
\item Take the predetermined coefficients $\vec{\mathbf{\alpha}}$ and $\vec{\mathbf{\beta}} $, which are generated outside the data generating process to provide consistency.  In this study, the values were as follows, 
\begin{align*}   
\begin{tabular}{L|L}
\; \; \; \;\; \; \; \;\; \; \; \; \vec{\mathbf{\alpha}} &\; \; \; \;\; \; \; \; \; \; \; \;  \vec{\mathbf{\beta}} \\ 
\hline 
\alpha_0 = 0.58986656 & \beta_0 = 0.17868818\\ 
\alpha_1 = 0.95344212 & \beta_1 = 0.89069712 \\ 
\alpha_2 = -0.89822429 & \beta_2 =   0.89037635 \\ 
\alpha_3 =  -0.95566697 & \beta_3 = 0.20497534 \\ 
\alpha_4 = 0.67520365 & \beta_4 =  0.10442911 \\ 
 \alpha_5 = 2.46365403 & 
\end{tabular} 
\end{align*}  
These coefficients were created by pulling each from $\vec{\mathbf{\alpha}} \sim Uniform(-1.0, 1.0)$ and $\vec{\mathbf{\beta}} \sim Uniform(-1.0, 1.0)$.  The one change made was that $\alpha_5$ had 1.5 added to the randomly generated value to ensure that the underlying covariate had significant impact.  

\item Create the underlying confounder, $U_i$ from $U_i \sim Unif(0.1, 1)$ 
\item Using the following logistic expressions, probabilities for each $L_{k,i}$ and $A_{k,i}$ can be obtained where $k$ is the time and $i$ is the individual, conditional on $\overline{L}_{k-1,i}$ and $\overline{A}_{k-1,i}$.  These probabilities are then used to obtain values for $L_{k,i}$ and $A_{k,i}$ using a binomial distribution with the respective probabilities.  
\begin{align} 
logitP[L_{k,i}] &= \alpha_0 + \alpha_1 \cdot L_{k-1,i} + \alpha_2 \cdot L_{k-2,i} + \alpha_3 A_{k-1,i} + \alpha_4 A_{k-2,i} + \alpha_5 U_i\\ 
logit[A_{k,i}] &= \beta_0 + \beta_1 L_{k,i} + \beta_2 L_{k-1,i} + \beta_3 A_{k-1,i} + \beta_4 A_{k-2,i}
\end{align} 
Note that for low time values when history is limited, the above expressions are slightly modified as follows, 
\begin{align} 
logitP[L_{0,i}] &= \alpha_0 + \alpha_5 U_i\\ 
logit[A_{0,i}] &= \beta_0 + \beta_1 L_{k,i} \\
logitP[L_{1,i}] &= \alpha_0 + \alpha_1 \cdot L_{k-1,i} + \alpha_3 A_{k-1,i}  + \alpha_5 U_i\\ 
logit[A_{1,i}] &= \beta_0 + \beta_1 L_{k,i} + \beta_2 L_{k-1,i} + \beta_3 A_{k-1,i} 
\end{align} 
\item Obtain a final $Y_i$ value for each individual where $Y_i \sim \mathcal{N}(\mu = U, \sigma = 1)$ 
\end{enumerate} 




%% Testing under the null of full treatment, but using data on time varying treatment 
\section{Parametric G-formula} 
Similar to IP weighting, parametric estimates can be obtained for standardized estimates.  An efficient method for doing this is the generalization of standardization to time-varying treatments and confounders, coined the g-formula method by Robins in 1986.\cite{robins1986new, wright2015international, hernan_robins_2016}  The method can be used for fixed and time-varying treatments in longitudinal studies, and it seeks to estimate the average causal effect of treatment.  

%%finish this sentence/thought below here 
which can be estimated as 
\begin{align} 
\mathbb{E}[Y^{\bar{a} = \bar{1}}] - \mathbb{E}[Y^{\bar{a} = \bar{0}}] 
\end{align} 
where the respective $\bar{a} = \bar{1}$ and $\bar{a} = \bar{0}$ signify constant treatment and no treatment over the entire time period.  

The g-formula seeks to calculate each standardized mean using the following, 
\begin{align} \label{eq:3} 
\mathbb{E}[Y^{\bar{a}= \bar{1}}] &= \sum_{l_i} \mathbb{E} \big[Y \mid  \overline{L}_{t}, \overline{A}_{t} \big]\cdot \prod_{k=0}^t P[L_k = l_k \mid \overline{L}_{k-1}, \overline{A}_{k-1}]
\end{align}
where $\overline{L}_k = \{L_{0} = l_0, L_{1} = l_1,  \cdots, \; L_{k} = l_k\}$ and $\overline{A}_k = \{a_{0} = 1, a_{1} = 1,  \cdots, \; a_{k} = 1\}$ are the history of the treatment and covariate variables up to and including time $k$.  The equivalent formula can be derived for $ \mathbb{E}[Y^{\bar{a} = \bar{0}}]$.   In expression \ref{eq:3}, the summation term is %% finish this sentence
%% come back here to discuss more about why you would use this method, how it was developed, details from Jamie's original paper, etc

One of the key reasons for using the g-formula method is that it is able to account for time-varying confounders which have feedback to each other.  This is equivalent to each $L_k$ being dependent on $A_{k-1}$.\cite{robins1986new}  In these scenarios, traditional methods for adjusting for the confounder, such as stratification, regression, and matching, may introduce bias; however, the g-formula method (as well as IP weighting) will not.\cite{wright2015international}  This is because these other methods are unable to estimate the joint effect of all treatment values $\{A_0, A_1 \dots A_K \}$ simultaneously and without bias.\cite{fitzmaurice2008longitudinal}  

The g-formula method has been shown to have a smaller variance than IP weighting methods, but this comes with added parametric modeling assumptions.\cite{young2011comparative} The smaller variance is due to the fact that the g-formula uses maximum likelihood estimates, in comparioson to the semi-parametic estimator used in IP weighting. Furthermore, IP weighting does fault and become quite unstable under violations (or close violations) of the positivity assumption, due to division by a near zero probability $P[A_k=a_k \mid \overline{L}_k, \overline{A}_{k-1}]$.  

These improvements are, however, under the assumption of exchangeability, and the fact that the g-formula relies more heavily on parametric assumptions, which can lead to bias.  The presence of bias is dependent on the accuracy of the models for $Y$, $A_k$ and $L_k$ for all $k$.  IP weighting methods are also dependent on the accuracy of their models, just different models such as for $A_k$ conditional on $\overline{L}_k, \overline{A}_{k-1}$.  

%% NEED TO LOOK INTO THE G-NULL PARADOX 
%% "theorem which implies that it can be essentially impossible to specify correct parametric models under the causal null hypothesis (i.e. the true risk difference under any two choices of x is exactly zero).  As a consequence, the method will reject the causal null even when true in sufficiently large samples." \cite{young2011comparative}

\subsection{Protocol} 
The method is performed in several steps, as follows 
%% SHOULD I REWRITE THIS AS NOT ABOUT EXPANDING THE DATASETS 
\begin{enumerate}  

\item \underline{Create outcome models}: Create models for the outcome variable $Y$ and the covariates, $L_k$ at each time using the original dataset.  The model for $Y$ is regressed on the treatment variable $A$ and the confounders, $L$. 

In this case, the following models were chosen for $Y \mid  \overline{A}_t, \overline{L}_t$ and $L_k \mid \overline{L}_{k-1}, \overline{A}_{k-1}$, 
\begin{align} 
\mathbb{E} \big[Y \mid \overline{A}_t, \overline{L}_t \big] &= \theta_{0} + \theta_1 A_{t} + \cdots + \theta_j A_0 + \theta_{j+1} L_t + \cdots + \theta_{j+k} L_0 \label{eq:2} \\ 
logit[L_k \mid \overline{L}_{k-1}, \overline{A}_{k-1}] &= \gamma_0 + \gamma_1 L_{k-1} + \gamma_2 L_{k-2} + \gamma_3 L_{k-3}  + \gamma_4 A_{k-1} + \gamma_5 A_{k-2} + \gamma_6 A_{k-3} \label{eq:3} 
\end{align} 
A time lag of only three historical values was deemed sufficient for the model of $L_k$ because ... 
%% PUT SOME REASONS IN HERE DISCUSSING HOW THESE MODELS WERE CHOSEN... SHOULD I BE USING QUADRATIC TERMS OR INTERACTION TERMS??!?!?!?!


\vspace{1cm}
Note that for initial time points where there was insufficient history for the full model, smaller models were created as follows 
\begin{align} 
Logit[L_1 \mid L_0, A_0]  &= \gamma'_0 + \gamma'_1 L_0 +  \gamma'_2 A_0 \label{eq:4} \\
Logit[L_2 \mid L_0, L_1, A_0, A_1] &= \gamma''_0 + \gamma''_1 L_{k-1} + \gamma''_2 L_{k-2}   + \gamma''_3 A_{k-1} + \gamma''_4 A_{k-2} \label{eq:5}
\end{align}


\item \underline{Predict using Monte Carlo}: Using the model created in step 2, predict the outcome $Y$ for the two new datasets created in step 1, conditioned on the given $A$ and $L$ values.  

Using expressions \ref{eq:2} through \ref{eq:5}, a Monte Carlo simulation must be performed to gain prediction values.  This is because it is impractical to calculate expression \ref{eq:3} directly for a continuous $L$.  This process is done as follows for time $k = \{ 0, \dots, K \}$, and $i = \{ 1, \dots, n \}$ keeping the test treatment regimen of interest $\bar{a}$ in mind through the process.  
\begin{enumerate} 
\item Select the $L_0$ value from a random individual from $i \in  \{ 1, \dots, n \}$. \label{step1}  
\item Obtain a probability of $L_1$ using this $L_0$ and $a_0$ in expression \ref{eq:4} and then obtain a sample $L_1$ by pulling from a binomial distribution.  
\item Obtain a probability of $L_2$ using the $L_0, L_1, a_0$, and $a_1$ in expression \ref{eq:5} and then obtain a sample $L_2$ by pulling from a binomial distribution.  
\item Continue the above process until time $K$ using expression \ref{eq:3} to get a full history $\overline{L}_K$ and all the probabilities $P[L_k = l_k \mid  \overline{L}_{k-1}, \overline{A}_{k-1}]$ 
\item Using expression \ref{eq:2}, $\bar{a}$ and the above solved for $\overline{L}_K$, calculate $\mathbb{E} \big[Y \mid \overline{A}_t, \overline{L}_t \big]$ 
\item Take the product of all the probabilities $P[L_k = l_k \mid  \overline{L}_{k-1}, \overline{A}_{k-1}]$ for $k = 0, \dots, K$ and $\mathbb{E} \big[Y \mid \overline{A}_t, \overline{L}_t \big]$ to get a final estimate.  \label{laststep}
\item Repeat steps (\ref{step1}) through (\ref{laststep}) for as many simulations as desired.  In this study, 10,000 individuals were simulated.    
\item Take the mean of all simulation values to obtain $\mathbb{E}[Y^{\bar{a}}]$ 
\item Repeat all above steps for the opposing treatment regimen of interest $\bar{a}'$ and take difference $\mathbb{E}[Y^{\bar{a}}] - \mathbb{E}[Y^{\bar{a}'}]$ to get the average causal treatment effect.  
\end{enumerate}
\end{enumerate} 

To do this process using non-parametric estimates, the Monte Carlo simulation is unnecessary.  Instead, the methodology would be to create two new simulated datasets, the first having all individuals under no treatment ($A=0$) and the second having all individuals treated ($A=1$).  Each of these new datasets has the same size as the original and the same ``individuals'', just changed values for $A$.  For these two datasets, delete the outcome values for $Y$ to treat it as a missing data.  Then, the outcome models would be calculated as above.  The prediction step, however, would differ, instead using the models directly to get predicted values for each individual in the extra two datasets.  Finally, the standardized means could be obtained by creating a weighted average for $E[Y^{a=0}]$ from the first new dataset and one for $E[Y^{a=1}]$ from the second new dataset.  

\subsubsection{Variance Estimate} 


%% TALK THROUGH HOW THIS IS DONE USING BOOTSTRAPPING 


\section{Doubly Robust Estimation} 
The method of doubly robust estimation, as proposed by Bang and Robins \cite{bang2005doubly}, combines the two previously discussed methods of IP weighting and standardization.  

IP weighting and standardization techniques are expected to provide different answers, unless there are no models used to create estimates.\cite{hernan_robins_2016}  
IP weighting estimates $P[A=a \mid L =l]$ using $P[A =a \mid L= l]$, while standardization estimates $E[Y \mid A = a, L=l]$  

- does not use the observed data treatment density \\
- estimators are consistent if either the model for treatment given the past (as in IP weighting) is correctly specified or the models for the outcome and covariates given the past (as needed to implement the parametric g-formula) are correctly specified, without knowing which is correct\\
- 

\subsection{Protocol} 
The method can be performed recursively using the following steps, 

\begin{enumerate}
\item Build a model for the treatment $A_k$ with data pooled for all time $m \in \{1, \dots, K \}$ and all individuals $i \in \{1, \dots n\}$ and obtain the MLE $\hat{\mathbf{\alpha}}$ of $\mathbf{\alpha}$ using logistic regression. 
\begin{align} 
logit\{P[A_{m,i} = 1 \mid \overline{l}_{m,i}, \overline{a}_{m-1,i}; \mathbf{\alpha}]\} &= w_m [\overline{l}_{m,i}, \overline{a}_{m-1,i}; \mathbf{\alpha}]
\end{align} 

This model can be rewritten as the following.  
\begin{align} 
f(A_m \mid \overline{L}_m, \overline{A}_{m-1}; \hat{\mathbf{\alpha}}) = \alpha_{0} + \alpha_{1} \cdot L_{m} + \alpha_{2} \cdot A_{m-1} + \alpha_{3} \cdot L_{m-1} + \alpha_{4} \cdot L_{m-2} + \alpha_{5} \cdot A_{m-2} 
\end{align} 

\item Set $\hat{T}_{K+1} = Y$ 

\item Recurse for $m = K+1, \dots, 2$ 
\begin{enumerate}
\item \label{modcreate} Use IRLS and a specified parametric regression model to get
\begin{align}
h_{m-1}(\overline{L}_{m-1}, \overline{A}_{m-1}; \mathbf{\beta}_{m-1}, \phi_{m-1}) = \Psi \{s_{m-1}(\overline{L}_{m-1}, \overline{A}_{m-1}; \mathbf{\beta}_{m-1}) + \phi_{m-1} \overline{\pi}_{m-1}^{-1} (\hat{\mathbf{\alpha}}) \}
\end{align}
which gives the conditional expectation of 
\begin{align}
\mathbb{E} \bigg[\hat{T}_m \mid \overline{L}_{m-1}, \overline{A}_{m-1} \bigg]
\end{align} 
The known function $s_{m}$ is specified on a case by case basis, and in this case was chosen to be as follows for the unknown parameter $\mathbf{\beta}$.  
\begin{align}
s_{m}(\overline{L}_{m}, \overline{A}_{m};\mathbf{\beta}_{m}) = \theta_0 + \theta_1 L_{m} +\theta_2 A_{m} + \theta_3 L_{m-1} +\theta_4 A_{m-1}+ \theta_5 L_{m-2} +\theta_6 A_{m-2}
\end{align}
Furthermore, the function $\overline{\pi}_m(\hat{\mathbf{\alpha}})$ is the propensity score model and is specified as follows
\begin{align} 
\overline{\pi}_m(\hat{\mathbf{\alpha}}) &= \prod_{j=1}^m f(A_m \mid \overline{L}_m, \overline{A}_{m-1}; \hat{\mathbf{\alpha}}) \\
&= \zeta_0 + \zeta_1 L_m + \zeta_2 L_{m-1} + \zeta_3 A_{m-1} + \zeta_4 A_{m-2}
\end{align}
The given $\Psi$ is the canonical link function of the chosen GLM.  \footnote{The desired method to do this is using a GLM with an underlying distribution (or family) of a Gaussian normal and a logit link.  However, python does not have the capacity to do it in this way, so alternatives had to be tested and considered, including basic linear regression with an expit applied after step \ref{} as well as using a logistic regression and taking the predicted probability to pull 1000 samples for each individual and regressing off that new data in the next step.}
%% COME BACK HERE AND WRITE IN WHAT I END UP CHOOSING

\item Let $ \hat{h}_{m-1}(\overline{L}_{m-1}, \overline{A}_{m-1}; \hat{\mathbf{\beta}}_{m-1}, \hat{\phi}_{m-1})$ be the predicted model derived in step \ref{modcreate}.  This implies that $(\hat{\mathbf{\beta}}'_{m-1}, \hat{\phi}_{m-1}')$ is a solution of 
\begin{align}
\mathbf{0} = \tilde{\mathbb{E}} \left[ \left[\hat{T}_{m} - \Psi \{ s_{m-1}(\overline{L}_{m-1}, \overline{A}_{m-1}; \hat{\mathbf{\beta}}_{m-1}) + \hat{\phi}_{m-1} \overline{\pi}^{-1}_{m-1}(\hat{\mathbf{\alpha}}) \} \right] \left( \frac{\partial s (\overline{L}_{m-1}; \mathbf{\beta}_{m-1})}{\partial \mathbf{\beta}'_{m-1}, \overline{\pi}^{-1}_{m-1}(\hat{\mathbf{\alpha}})} \right) \right]
\end{align}
where $\tilde{\mathbb{E}}(X) = \frac{1}{n} \sum_{i=1}^n X_i$ is the computational average.  

\item Set 
\begin{align} 
\hat{T}_{m-1}^{a_{m-1}, \dots, a_K} &= \hat{h}_{m-1}(\overline{L}_{m-1}, \overline{A}_{m-2}, a_{m-1}) \\
&=  \Psi \{s_{m-1}(\overline{L}_{m-1}, \overline{A}_{m-2}, a_{m-1}; \mathbf{\beta}_{m-1}) + \phi_{m-1} \overline{\pi}_{m-2}^{-1} (\hat{\mathbf{\alpha}}) f(a_{m-1} \mid \overline{L}_{m-1}, \overline{A}_{m-2}; \hat{\mathbf{\alpha}}) \}
\end{align} 
where $a_{m-1}$ is our treatment value of interest, the lowercase letter indicating a test value rather than an observed.  
\end{enumerate}

\item To calculate the final $\mathbb{E}[Y^{\bar{a}}]$, solve 
\begin{align} 
\mathbb{E}[Y^{\bar{a}}] = \tilde{\mathbb{E}}(\hat{T}_1) = \tilde{\mathbb{E}}(\hat{T}_1^{\bar{a}})
\end{align}

\item Repeat all above steps for the opposing treatment regimen of interest $\bar{a}'$ and take difference $\mathbb{E}[Y^{\bar{a}}] - \mathbb{E}[Y^{\bar{a}'}]$ to get the average causal treatment effect.  
\end{enumerate}




