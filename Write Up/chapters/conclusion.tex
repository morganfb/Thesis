%!TEX root = ../dissertation.tex
\chapter{Conclusion}
\label{conclusion}


% \section{Summary} 
\newthought{In conclusion}, two methods of estimating causal treatment effect are presented here: g-formula and doubly robust estimation.  The evidence presented here leads to the conclusion that the method of doubly robust estimation could aptly be renamed multiply robust estimation, as it can robustly withstand substantial variation and misspecification in model selection.  The doubly robust estimator is perhaps not as precise as the g-formula, but it is more robust and much more efficient, as well as far more accessible to the average user. 


\section{Contributions}
These methods have never been implemented in this form in Python before, an advance that will significantly increase the efficiency and accessibility of use, decreasing barriers between scientific study and causal inference.  The functions currently created and listed in Appendix \ref{AppendixA} could be easily restructured in order to be widely used and implemented.  Capability must also be extended for including an extensive number of covariates.  


\section{Implications}
The evidence presented here regarding the robustness of the doubly robust method and its increased efficiency are likely to have an impact on many fields of science and medicine.  It could be efficiently applied to massive datasets, often used in medical research in pursuit of FDA approval, possibly proving many novel discoveries.  This application alone could potentially save the field of medicine millions, even billions, of dollars in years to come.   

\section{Limitations}
Both methods studied require a stringent set of assumptions, many of which are difficult to confirm

\section{Extensions} 