%!TEX root = ../dissertation.tex
\chapter{Conclusion}
\label{conclusion}


% \section{Summary} 
\newthought{In conclusion}, two methods of estimating causal treatment effect are presented here: the g-formula and the doubly robust estimators.  The evidence presented leads to the conclusion that the method of doubly robust estimation could aptly be renamed multiply robust estimation, as it can robustly withstand substantial variation and misspecification in model selection.  The doubly robust estimator is perhaps not as precise per iteration as the g-formula under the null hypothesis, but it is more robust to user model misspecification and is more precise on average.  Furthermore, although the bias is unmeasurable under the alternative hypothesis, the doubly robust shows no pressing issues such as those seen with the g-formula under the alternative hypothesis.  

\section{Contributions}
These methods have never been implemented in this form in Python before, an advance that will significantly increase the efficiency and accessibility of use, decreasing barriers between scientific study and causal inference.  The functions currently created and listed in Appendix \ref{AppendixA} could be easily restructured into a software package in order to be widely used and implemented.  Capability must also be extended more effectively for including an extensive number of covariates, an important assumption of the model.  

The Python implementations make both methods highly efficient, but the difference seen is more significant for the g-formula.  The doubly robust estimator has previously been considered a more efficient means of estimating treatment effects because of its recursive nature, in comparison to the g-formula which has suffered inefficiencies due to the high number of replicates required for the Monte Carlo simulation.  Of note, the parallelization of the Monte Carlo simulation in this implementation actually improves efficiency significantly, making the efficiency difference between the two methods less concerning.  

This thesis first confirmed that the doubly robust estimator is indeed doubly robust, meaning that as long as at least one of the two models contained within is correctly specified, the treatment effect remains unbiased.  Furthermore, this thesis showed that the doubly robust estimator is actually more robust than thought, remaining unbiased as long as the $\pi$ models are correctly specified before the $s$ models chronologically.  

In total, the two methods were implemented in Python, and their results compared across several criteria.  The g-formula is quicker, but prone to error from user misspecification and problems under the alternative hypothesis of a treatment effect.  On the other hand, the doubly robust estimator is more accurate albeit with slightly higher variance across simulations, but it is significantly more robust to situations that could have the potential to introduce bias, such as slight user misspecification or in the scenario of a treatment effect.  

\section{Limitations}
Both methods studied require a stringent set of assumptions, many of which are difficult to confirm.  Because of this, finding the right situation in which to use these methods can be difficult.  Many of the assumptions can be approximately achieved, although the inability to confirm their fulfillment makes it difficult to calculate the bias of these approximations.  

\section{Implications}
The evidence presented here regarding the increased efficiency of the two methods' implementation in Python is likely to have an impact on many fields of science and medicine.  This implementation could be efficiently applied to massive datasets, often used in medical research in pursuit of FDA approval, possibly proving many novel discoveries.  These methods can provide significant utility for the astonishing amounts of data that is not frequently capitalized upon.  Even if the assumptions are difficult to confirm, it would serve as a point of positive progress towards effective use of all of this data.  As a result, these methods would provide the framework for impactful changes in the field of medicine, such as fewer clinical trials, more discoveries, and faster innovation.  This application alone could potentially save the fields of medicine and healthcare millions, even billions, of dollars in years to come. 

Furthermore, the robustness of the doubly robust method presents it as an improved estimator for research purposes because it can sustain substantial errors caused by the researcher.  The method can withstand significant model misspecification without introducing bias, meaning that researchers do not need perfect statistical knowledge in order to use this method.  This makes it more accessible to users of all backgrounds.  No longer is a PhD in statistics required to estimate causal effect.  This implementation and its improved robustness makes the method much more accessible to research across industries and subjects.  

Finally, in addition to the implications for observational data, this method is an important development in the study of sequentially randomized trials.  These trials present as very difficult to implement for researchers due to their complicated design, but these methods make them much simpler to study and draw results from.  Although not as widely used as traditional clinically randomized trials, this type of trial is likely to become more widespread in the future, requiring the use of more complex analysis techniques, such as the two discussed herein.  

\section{Extensions}
This thesis focuses on the framework of medical research in order to contextualize the methods; however, these methods can be applied to data from any field.  It would likely be of particular interest in the social sciences, finance, and other fields of science.  Limited only by the assumptions, this method can be applied to large datasets to estimate causal effects, leading to the advancement of research across many fields.  With further research and work on the program, these methods could be easily accessible by users from many fields.  

