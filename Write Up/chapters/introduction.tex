%!TEX root = ../dissertation.tex
\chapter{Introduction}
\label{introduction}

\newthought{Causation versus correlation} - the age-old debate rages on among statisticians, scientists, and students alike.  Correlation is the easier idea to understand: are two or more things related to each other?  Do taller people weigh more than shorter people?  Is the temperature colder when there is snow on the ground than when there is not?  Do people who drink red wine and eat dark chocolate have healthier hearts? 

Proving correlation is simpler than causation.  Select a group of random, unrelated people who drink red wine and eat dark chocolate and a second group of random, unrelated people who do not, and compare their resting heart rates and HDL (``good'' cholesterol) levels.  If the red wine drinkers and chocolate consumers have better heart health, then it can be concluded that consuming red wine and dark chocolate is associated with a healthier heart.  But, there is a catch.  This exercise only demonstrates that consuming red wine and chocolate is correlated with a healthy heart, but it does not prove that red wine and chocolate actually cause a heart health.  Perhaps everyone who consumes red wine and dark chocolate also exercises more frequently, is of a healthier weight, or is younger than those who do not, all possible factors that could lead to a healthier heart separate from an individual's red wine drinking and chocolate eating habits.  The association found between dark chocolate and red wine consumption could be reflective of these other healthy heart habits rather than an indication of red wine and dark chocolate being the driver of heart health.  This exercise of association does not isolated the underlying cause of the difference. 
 
This begs the question of causation; how one can actually prove that the red wine and the dark chocolate cause heart health.  Answering this question is not quite as simple as showing correlation, and traditionally, a randomized trial is required.  A randomized trial experiment involves enrolling a sufficient number of patients, who are randomly assigned into one of two groups: those who are told to drink red wine and eat dark chocolate (the treatment group) and those who are told not to (the controls).  A trial like this would likely go on for some extended period of time, maybe years.  The individuals in both groups would have their heart rate and their HDL levels measured along the way.  Using this data, the researchers could compare the heart health of the two groups and come to a definitive conclusion (pending statistical significance) of whether drinking red wine and eating dark chocolate caused a healthier heart.  By quantifying the difference in outcome between treatment and control groups, researchers can demonstrate an association between treatment and outcome.  The randomized nature of the trial is what allows researchers to conclude that in these two groups, the association consuming red wine and dark chocolate is actually causation.\footnote{This idea is discussed further in Section \ref{assumptions}.} However, this process of the controlled randomized trial is complicated, costly, and until recently, the only option to demonstrate causation.  
 
Randomized controlled trials are commonly used in medicine to test the efficacy of a treatment drug.  In the US, it takes 12 years and an average of more than \$350 million USD on average to undergo drug testing and FDA approval.\footnote{New Drug Approval Process, \url{https://www.drugs.com/fda-approval-process.html}}  FDA approval requires proof of statistically significant causation of the drug's efficacy through at least two randomized controlled trials.  The difficulty surrounding these trials makes the medical innovation process terribly slow, expensive, and inefficient, purely because the current means of proving causation is so challenging.  
 
Further complicating this process, a more complex randomized control trial exists in the form of the sequentially randomized trial.  In these trials, several groups of patients exist, rather than just the control and the treatment groups.  Furthermore, the random mechanism for assigning treatment or control typically varies by group.  For instance, consider a trial studying a drug to treat ovarian cancer, where the researchers want to test the efficacy of the drug on different stages of the cancer and want to use a more intense regimen for those with life threatening disease.  The researchers decide to only accept subjects whose cancer is in stage 1 or 2, and for those individuals at each time point, the decision to assign treatment or control is entirely random.  However, as the trial continues, some subject's cancer progresses to stage 3 or 4, so the doctors want to increase the drug dosage.  For those subjects, researchers again randomly assign subjects to a different more aggressive treatment or to the same regimen.  Because the prognosis is more pressing at this point, the researchers may decide that the likelihood of switching treatments should be higher in their random assignment method than the likelihood of maintaining treatment.  At each checkup, the researchers reassess whether an individual should be switched from the less aggressive treatment group to the more aggressive treatment group and from there randomly assign a treatment regimen, which can be no treatment.  

A sequentially randomized trial such as this one is practically more effective and hopefully provides a more nuanced understanding of a treatment's efficacy.  However, this comes at the cost of being quite complicated to analytically draw conclusions from.  No longer can researchers perform a simple test on an outcome measure between the control group and the treatment group.  Now, there are many more than two groups and the treatments which vary over time make this even more complicated.  
 
However, in the last few decades, a field of statistics known as causal inference has emerged.  It creates and studies various methods to attempt to prove causation by means other than the randomized trial.  This field emphasizes methods that can work strictly from observational data, or data that is already collected and does not need to be controlled or dictated by the investigator.  Some examples of observational data include hospital data accumulated over millions of patients and years that track symptoms, level types, outcomes, and demographic surveys, as well as data from other experimental trials which is now being reconsidered for a different question of interest.  Using these alternative methods of determining causality, drug efficacy could be proven with significantly less cost and time, impacting millions of lives.  

This thesis is a comparative examination of two such methods of causal inference: the g-formula and the doubly robust estimator.  Both methods seek to prove from purely observational data the same causal effect as a randomized trial. They also have the capacity to do so for dynamic treatment regimens, such as those seen in the sequentially randomized trial, an improvement not seen in former methods of causal inference.  The doubly robust estimator is an improved development on the g-formula and is shown to be much more effective at correctly approximating causal effect.  It will be shown that this method can withstand significant error caused by human inaccuracy in model selection, leading to a better estimator.  In short, this more advanced method is more efficient, effective, and user-friendly, paving the road for widespread future use in medicine as novel discoveries appear from decades-worth of underutilized data.  


% Randomized control trials are a common occurrence in the field of medicine, comes from the field of medicine, where researchers can perform a controlled experiment to prove causation.  A controlled study typically contains two groups, one that receives no treatment (the placebo group) and one that receives the treatment (the treatment group).  Individuals are randomly assigned into each group, and by comparing the outcome of these two groups, the researchers can demonstrate whether the outcome for patients receiving treatment differs significantly from the controls.    For instance, clinical trials are often conducted to study the efficacy of cancer therapies, prescribing treatment to individuals randomly assigned to the treatment group and not to those assigned to the control group.  Outcome can easily be quantified by metrics such as survival, prevalence of cancer cells and %INSERT WORD HERE.  


 
 
 
 



