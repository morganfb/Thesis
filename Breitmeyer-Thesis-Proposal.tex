\documentclass[a4paper]{article}

\usepackage[english]{babel}
\usepackage[utf8]{inputenc}
\usepackage{amsmath}
\usepackage{graphicx}
\usepackage[colorinlistoftodos]{todonotes}
\usepackage{cite}
\usepackage{caption}
\usepackage{subcaption}

\title{Evaluating Causal Inference Techniques Using the \\ g-formula in Two Different Implementations }

\author{Morgan F. Breitmeyer}

\date{\today}

\begin{document}
\maketitle

\section{Research Questions of Interest} 
In the realm of observational data, causal inference is frequently used to identify causal relationships.  The goal is to effectively simulate a randomized trial experiment retroactively.  The most widely used method for doing this is through using the g-formula, which parametrically estimates an average casual effect.  This method depends on finding parametric models, which are marginal distributions of covariates in the data which can be empirically estimated.  Subsequently, a Monte Carlo simulation is used on the estimated parameters so estimates of the causal effects can be measured.  

The goal of the thesis is to implement a sequential version of the g-formula as a proof of concept.  This has yet to be done in a sophisticated manner and will hopefully be scalable for wider research.  In order to prove its efficacy, it will be applied to a dataset of interest, likely something of the epidemiological sort.  The two implementations of the g-formula will be compared on this dataset and hopefully conclusions will be drawn.  The biggest questions are whether implementation of this version of the g-formula is possible, whether it will be successful, and if it's more efficient than the existing implementation such that it could be scalable to large data sets.    

I will be advised by Dave Harrington and Miguel Hernan (and perhaps Jamie Robbins in a less official capacity).  

\section{Outline} 
\begin{itemize} 
\item November 10: Literature review and developing a deep understanding of causal inference and g-formulas.  This will include building out why causal inference is of interest, the intuition behind both implementations of the g-formula, and outlining the limitations and assumptions of these methods.  This part of the process may take longer than just November, but will at least be a significant start.  
\item End of 2016/early 2017: Hopefully, I will have the implementation of the formula almost complete.  I will be using python to do this.  
\item January 2017: Do analysis of a dataset using the two methods. Building out the two models and their likelihood models will likely take quite a bit of time so I want to leave room in here for this.  
\item February/March 2017: Wrapping everything up and explaining all of the analyses and how the implementations worked/differed/etc.  
\end{itemize} 

\nocite{*} 

\newpage
\bibliography{Bibliography}
\bibliographystyle{plain}

\end{document}